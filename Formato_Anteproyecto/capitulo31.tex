\chapter{Fundamentos Metodol\'ogicos}
	
	En este capítulo se detalla el enfoque, tipo, nivel, diseño de la investigación y metodología a implementar como estructura a seguir por el trabajo presente. 
	
\section{Enfoque de la investigaci\'on}

	Esta investigación posee un enfoque cuantitativo, que como lo indica Sampieri et al (2014) es  secuencial y probatorio, cada etapa precede a la siguiente y no podemos “brincar” o eludir pasos. El orden es riguroso, aunque desde luego, se puede redefinir alguna fase. Parte de una idea que va acotándose y, una vez delimitada, se derivan objetivos y preguntas de investigación, se revisa la literatura y se construye un marco o una perspectiva teórica. De las preguntas se establecen hipótesis y determinan variables; se traza un plan para probarlas (diseño); se miden las variables en un determinado contexto; se analizan las mediciones obtenidas utilizando métodos estadísticos, y se extrae una serie de conclusiones.\\
	
	
\section{Tipo o nivel de investigaci\'on}

Según Pallela y Martins (2012) la investigación de campo consiste en la recolección directamente de la realidad donde ocurren los hechos, sin manipular o controlar variables. Este proyecto plantea ese tipo de investigación ya que busca explorar los efectos de la interrelación entre los diferentes tipos de variables en lugar de los hechos.\\

El nivel de investigación es correlacional, porque se quiere conocer la relación o grado de asociación que existe entre dos o más conceptos, categorías o variables en una muestra o contexto particular, es decir, se busca asociar variables mediante un patrón predecible para un grupo. (Sampieri et al, 2014).

	
\section{Dise\~no de la investigaci\'on}

El término diseño se refiere al plan o estrategia concebida para obtener la información que se desea con el fin de responder al
planteamiento del problema (Sampieri et al, 2014), el diseño de la investigación presente es no experimental cuantitativa porque como lo indica Sampieri et al (2014) “es la investigación que se realiza sin manipular deliberadamente variables. Es decir, se trata de estudios en los que no hacemos variar en forma intencional las variables independientes para ver su efecto sobre otras variables. Lo que hacemos en la investigación no experimental es observar fenómenos tal como se dan en su contexto natural, para analizarlos”.\\

De acuerdo al objetivo principal planteado por esta investigación se busca clasificar la densidad de papa criolla a partir de sus pesos frescos, esto indica buscar la relación de estás variables, lo que conlleva a tener un diseño transeccional correlacional-causal que según Sampieri et al (2014) describen relaciones entre dos o más categorías, conceptos o variables en un momento determinado. A veces, únicamente en términos correlacionales, otras en función de la relación causaefecto (causales).
  

\section{Metodolog\'ia}

El desarrollo de la investigación en la que se propone la realización de un algoritmo que emplee clasificación de Bayes Naive comprende las siguientes fases metodológicas, según la metodología de CRISP-DM (\emph{Cross Indrustry Standard Process for Data Mining}):\\

\noindent
\textbf{Comprensión del problema}.\\

	Esta fase inicial se centra en la comprensión de los objetivos, requisitos y restricciones del proyecto desde una perspectiva no técnica, con el fin de convertirlos en objetivos técnicos y en un plan de proyecto. En esta fase, es muy importante la capacidadde poder convertir el conocimiento adquirido del problema en un plan preliminar cuya meta sea el alcanzar los objetivos del problema. Las principales tareas que componen esta fase son las siguientes:\\

(a).	Determinar los objetivos del problema.
(b).	Evaluación de la situación.
(c).	Determinación de los objetivos del proyecto propuesto.
(d).	Producción de un plan del proyecto.\\


\noindent
\textbf{Comprensión de datos}.\\

	La segunda fase comprende la recolección inicial de los  datos, con el objetivo de establecer un primer contacto con el problema, familiarizándose con ellos, identificar su calidad y establecer las relaciones más evidentes que permitan definir las primeras hipótesis, si fuera el caso. Las principales tareas a desarrollar en esta fase del proceso son:\\

(a).	Recolección de datos iniciales. 
(b).	Descripción de los datos.
(c).	Exploración de datos.
(d).	Verificación de la calidad de los datos.\\

\noindent
\textbf{Preparación de datos}.\\

En esta fase y una vez efectuada la recolección inicial de datos, se procede a su preparación para adaptarlos a las técnicas de  ......, tales como técnicas de visualización de datos, de búsqueda de relaciones entre variables u otras medidas para exploración de los datos. La preparación de datos incluye las tareas generales de selección de datos a los que se va a aplicar una determinada técnica de modelado, limpieza de datos, generación de variables adicionales, integración de diferentes orígenes de datos y cambios de formato. Las principales tareas involucradas en esta fase son las siguientes:\\

\begin{itemize}
\item	Selección de datos.
\item Limpieza de los datos.
\item	Estructuración de los datos.
\item	Integración de los datos.
\item	Formateo de los datos.\\
\end{itemize}

\noindent
\textbf{Modelado}.

En esta fase de CRISP-DM, se  selecciono la técnica  de modelado más apropiadas para el proyecto propuesto. ya que cumple con los siguientes criterios: 

\begin{itemize}
\item	Ser apropiada al problema. 
\item	Disponer de datos adecuados. 
\item	Cumplir los requisitos del problema. 
\item	Tiempo adecuado para obtener un modelo. 
\item	Conocimiento de la técnica.
\end{itemize}

Previamente al modelado de los datos, se debe determinar un método de evaluación de los modelos que permita establecer el grado de bondad de ellos para este caso la bondad de ajuste se evaluara mediante curvas ROC.\\\

\noindent
\textbf{Evaluación}.\\

	En esta fase se evalúa el modelo, teniendo en cuenta el cumplimiento de los criterios de éxito delproblema. Debe considerarse, además, que la fiabilidad calculada para el modelo se aplicasolamente para los datos sobre los que se realizó el análisis. Es preciso revisar el proceso,teniendo en cuenta los resultados obtenidos, para poder repetir algún paso anterior, en el que sehaya posiblemente cometido algún error.Las tareas involucradas enesta fase del proceso son las siguientes:\\

\begin{itemize}
\item	Evaluación de los resultados.
\item	Proceso de revisión.
\item	Determinación de futuras fases.
\end{itemize}

\section{Aspectos administrativos}

\vspace{1 cm}
La realización de la investigación será planificada según lo establecido en el siguente diagrama:\\

\begin{figure}[!ht]
\begin{center}

\begin{ganttchart}[y unit title=0.4cm,
y unit chart=0.5cm,
vgrid,hgrid,
title height=1,
bar/.style={draw,fill=cyan},
bar incomplete/.append style={fill=yellow!50},
bar height=0.7]{1}{16}
 \gantttitle{Semanas}{16}\\
 \gantttitle{Julio}{2}
 \gantttitle{Agosto}{5} 
\gantttitle{Septiembre}{4}
\gantttitle{Octubre}{5}\\
 \gantttitlelist{1,...,2}{1}
 \gantttitlelist{1,...,5}{1}
 \gantttitlelist{1,...,4}{1} 
 \gantttitlelist{1,...,5}{1} \\
 \ganttbar{\tiny{Comprensión del problema}}{1}{2} \\
 \ganttbar{\tiny{Comprensión de datos}}{3}{4} \\
 \ganttbar{\tiny{Preparación de datos}}{4}{6} \\
 \ganttbar{\tiny{Modelado}}{6}{13} \\
 \ganttbar{\tiny{Evaluación}}{9}{14} \\
%  \ganttbar[progress=70]{Fase 3}{13}{18} \\
 % \ganttbar[progress=40]{Conclus\~ao}{20}{24} \\
 \ganttbar{\tiny{Realización del informe del proyecto especial de grado}}{2}{16} \\
 \ganttlink{elem0}{elem1}
\end{ganttchart}

\end{center}
\caption{Diagrama de Gantt con la planficación del proyecto especial de grado}
\end{figure}


	