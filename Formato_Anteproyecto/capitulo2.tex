\chapter{Fundamentos te\'oricos}

\section{Antecedentes}

Internacional. Noor Amaleena Mohamad, Noorain Awang Jusoh, Zaw Zaw Htike y Shoon
Lei Win, 2014. Bacteria Identification from Microscopic Morphology using Naïve Bayes.\\

El objetivo de la investigación fue proponer un marco automatizado de identificación de bacterias que pudiera clasificar tres famosas clases de bacterias llamadas Cocci, Bacilli y Vibrio desde la morfología microscópica usando el clasificador Naïve Bayes, para desarrollar el marco se comprendieron dos fases, la primera fue el entrenamiento del sistema empleando un set de imagenes microscópicas que contenian Cocci, Bacilli y Vibrio, las imagenes de entrada fueron normalizadas para enfatizar el diametro y forma de las caracteristicas. En la segunda etapa se empleo el clasificador Naïve Bayes para realizar inferencia probabilística basada en los descriptores de entrada. Para el entrenamiendo se utilizaron 65 imagenes de cada clase de bateria, para las pruebas fueron usadas 222 imagenes que poseian las tres clases de bateria e imagenes aleatorias de humanos y aviones, durante las pruebas el sistema fue capaz de discriminar correctamente entre las tres clases de bacterias e incluso logró rechazar las imagenes que no pertenecian a ninguna de las tres clases de bacterias, como conclusión la investigación demostró como un simple clasificador con unas cuantas caracteristicas basadas en imagenes puede proveer una alta exactitud en la identificación de bacterias según su morfología microscópica, este marco de identificación que consiste en la extracción y clasificación ha logrado un 80\% de exactitud al clasificar las tres bacterias (Cocci, Bacilli y Vibrio), a pesar de su naturaleza exploratoria se considera que se debe realizar más trabajo para lograr una clasificación robusta y de mayor exactitud empleando aprendizaje automático no solo para bacterias sino para cualquier otro objeto clasificable.\\

Internacional. Misigo Ronald y Miriti Evans, 2016. Classification of Selected Apple Fruit Varieties using Naïve Bayes.\\

La necesidad de distinguir variedades de manzanas de una manera rápida y no destructiva motivó la investigación presente que tiene como objetivo principal investigar la aplicabilidad y el rendimiento del algoritmo de clasificación de Naïve Bayes para distinguir las variedades de manzanas, la metodología aplicada involucro la adquisición de las imagenes, preprocesamiento y segmentación, analisis y clasificación de las variedades de manzanas. Se realizo un muestreo aleatorio y se emplearon 60 imagenes para el entrenamiento del clasificador, 30 imagenes para la validacion y 60 imagenes para las pruebas, los resultados fueron positivos verdaderos, positivos falsos, negativos verdaderos y negativos falsos, luego se evaluó el rendimiento del sistema, obteniendo los valores estimados para exactitud, sensibilidad, precisión y especificidad donde se obtuvo 91\%, 77\%, 100\% y 80\% respectivamente, en conclusión la clasificación empleando Naïve Bayes resultó en un porcentaje mayor de exactitud que las técnicas de lógica difusa y MLP-Neural que habian sido empleadas previamente para realizar la tarea de clasificación.\\

Internacional. Arias Victoria, Bustos Patricia y Ñústez Carlos, 1996. Evaluación del Rendimiento en Papa Criolla (Solanum phureja) variedad "`Yema de Huevo"', bajo diferentes Densidades de Siembra en la Sabana de Bogotá.\\

Se evaluó el rendimiento de la papa criolla (Solanum phureja Juz. et Buk.) variedad "`yema de huevo"', bajo diferentes densidades de siembra, utilizando cuatro distancias entre surcos (0,70;0.80;0.90 y 1,0m), en Cundinamarca, Colombia. Las variables de rendimiento evaluadas fueron: peso y número de tubérculos de primera, segunda y tercera clase por metro cuadrado, y peso y número total de tubérculos por metro cuadrado. Las diferentes densidades evaluadas no presentaron diferencias significativas para el número y peso de tubérculos de primera y segunda clase, para las distancias entre surcos menores de un metro, se encontraron incrementos significativos en el peso total de tubérculos, pero se redujo el tamaño promedio de los mismos, es decir, que se obtuvo mayor número y peso de tubérculos de tercera clase.\\ 

Internacional. Paraskevas Tsangaratos y Ioanna Ilia, 2016. Comparison of a logistic regression and Naïve Bayes classifier in landslide susceptibility assessments: The influence of models complexity and training dataset size.\\ 

El objetivo principal de la investigación fue comparar el rendimiento de un clasificador que implementa regresión logística con uno que implementa el algoritmo de Bayes Naive en evaluaciones de susceptibilidad a deslizamientos. El estudio proporciona una evaluación sobre la influencia de la complejidad del modelo y el tamaño del conjunto de datos de entrenamiento, mientras que identifica el clasificador más preciso y confiable. La comparación de los dos clasificadores se basó en la evaluación de una base de datos que contiene 116 sitios ubicados en las montañas de Epiro, Grecia, donde se han encontrado eventos graves de derrumbes. Los sitios están clasificados en dos categorías, áreas sin deslizamientos de tierra y derrumbes. En particular, se analizaron siete variables: unidades geológicas de ingeniería, ángulo de la pendiente, aspecto de la pendiente, promedio anual de precipitación, distancia de la red fluvial, distancia de las características tectónicas y distancia de la red de carreteras. Se implementó el análisis de multicolinealidad y la selección de características para estimar la independencia condicional entre las variables y para clasificar las variables según su importancia en la estimación de la susceptibilidad al deslizamiento.\\

Mediante los procesos anteriores, se logró la construcción de nueve conjuntos de datos diferentes, promover la partición permitió crear subconjuntos de entrenamiento y validar datos de los 116 sitios originales. Cada conjunto de datos era caracterizado por el número de variables utilizadas y el tamaño de los conjuntos de datos de entrenamiento. La comparación y validación de los resultados de cada modelo se logró utilizando medidas de evaluación estadística, la característica operativa de recepción y el área bajo las curvas de éxito y velocidad predictiva. Los resultados indicaron que la complejidad del modelo y el tamaño del conjunto de datos de capacitación influyen en la precisión y la capacidad predictiva de los modelos concernientes a la susceptibilidad al deslizamiento. En particular, el modelo más preciso con alto poder predictivo fue el octavo modelo (cinco variables y 92 datos de entrenamiento), con el clasificador Bayes Naive teniendo un rendimiento y precisión generales ligeramente más altos que el clasificador de regresión logística, 87.50\% y 82.61\% en los conjuntos de datos de validación, respectivamente. El área más alta bajo la curva se logró mediante el clasificador Naïve Bayes para los conjuntos de datos de entrenamiento y validación (0.875 y 0.806 respectivamente) mientras que el clasificador de regresión logística logró valores de AUC más bajos para los conjuntos de datos de capacitación y validación (0,844 y 0,711, respectivamente). Cuando hay datos limitados disponibles, parece que se podrían obtener resultados más precisos y confiables mediante clasificadores generativos, como clasificadores Bayes Naive. 

\section{Bases Te\'oricas}

\subsection{Python}

Python es un lenguaje de programación de código abierto creado por Guido van Rossum. Una de las ideas claves de van Rossum era que los programadores pasaban más tiempo leyendo código que escribiendolo, entonces creo un lenguaje fácil de leer. Python es uno de los lenguajes de programación más populares y fáciles de aprender. Funciona en la mayoría de sistemas operativos y computadoras y es usado desde la construcción de servidores web hasta crear aplicaciones de escritorio. (Althoff,2016)

\subsection{PyCharm}
PyCharm es un entorno de desarrollo integrado dedicado a Python y Django que provee un amplio rango de herramientas esenciales para programadores, que están estrechamente integrados para crear un entorno conveniente para el desarrollo productivo de Python.

\subsection{Regresión Espacial}

El método de regresión espacial es un modelo estadístico para data observada en unidades geográficas como países o regiones, donde juegan un papel importante los vecinos como indica Arbia (2014), este autor expresa que para tratar información espacial es necesario tener dos sets de información, el primero que posee los valores observados de las variables y el segundo que posee la ubicación particular donde esos valores fueron observados y las relaciones de proximidad entre todas las observaciones espaciales. (hablar de autocorrelación espacial y supuestos de independencia).


\subsection{Clasificador Bayes Naive}

El clasificador Bayes Naive es un clasificador probabilístico basado en el teorema de Bayes, considerando la suposición de independencia ingenua. Los clasificadores Bayes Naive suponen que el efecto de un valor de variable en una clase dada es independiente de los valores de otra variable. Esta suposición se llama independencia condicional de clase. Bayes Naive a menudo puede realizar métodos de clasificación más sofisticados, es particularmente adecuado cuando la dimensionalidad de las entradas es alto. Cuando se quieren resultados más competentes, en comparación con otros métodos de salida, podemos usar la implementación de este clasificador que crea modelos con capacidades predictivas.
(Misigo y Miriti, 2016)

\subsection{Teorema de Bayes}

\subsection{Curva Característica Operativa del Receptor}

Una curva característica operativa del receptor es una herramienta estadística para evaluar la precisión de predicciones, a menudo se abrevia como curva ROC o gráfico ROC, el último se utiliza con más frecuencia en la literatura de minería de datos.(Gönen,2007)\\

Las curvas ROC proporcionan una forma completa y visualmente atractiva de resumir la precisión de las predicciones. Son ampliamente aplicables, independientemente de la fuente de predicciones. También puede comparar la precisión de los diferentes métodos de generación de predicciones al comparar las curvas ROC de las predicciones resultantes.(Gönen,2007)

 
\subsection{Glosario}

\paragraph{}: 


