\chapter{Preliminares}

\section{Planteamiento y formulaci\'on del problema}

Los algoritmos, la estadística, la ingeniería, la optimización y diversas ramas de la
ciencia computacional son parte de las herramientas que ayudan a describir un aspecto
del mundo real a través de un conjunto de datos,  que en muchos casos está conformado
por las muestras que son representaciones de objetos reales y las características que
son la descripción de dichos objetos. (Layton, 2015)\\

Según VanderPlas (2017) para realizar clasificación se pueden implementar los modelos
\textit{Bayes Naive} que son un grupo de algoritmos simples y rápidos mayormente
acertados para conjuntos de datos grandes, estos clasificadores están construidos sobre
la base de los métodos de clasificación Bayesiana que se apoyan en el teorema de Bayes
que describe la relación de las probabilidades condicionales de las cantidades estadísticas.
En la clasificación Bayesiana el interés es encontrar la probabilidad de una clase L de
acuerdo a las características observadas, que se puede escribir como
$P\left(L|caracteristicas\right)$ y el teorema de Bayes indica como expresar esto en
términos de cantidades que pueden ser computarizadas más fácilmente:

\[P\left(L|caracteristicas\right)=\frac{P\left(caracteristicas|L\right)P\left(L\right)}{P\left(caracteristicas\right)}\]

Si se intenta decidir entre dos clases ($L_{1}$ y $L_{2}$) la manera de tomar la decisión
es calcular el radio de la probabilidad posterior para cada clase:

\[\frac{P\left(L_{1}|caracteristicas\right)}{P\left(L_{1}|caracteristicas\right)}=\frac{P\left(caracteristicas|L_{1}\right)P\left(L_{1}\right)}{P\left(caracteristicas|L_{2}\right)P\left(L_{2}\right)}\]\\

El algoritmo basado en estas ecuaciones debe seguir los siguientes pasos según Layton
(2015):

\begin{itemize}
	\item Teniendo el conjunto de datos de entrenamiento se debe calcular los valores
	de la probabilidad de una característica para cada clase.
	\item Se computa la probabilidad de que un dato de muestra pertenezca a una clase.
	\item Se computa la probabilidad de que un dato pertenezca a una clase.
	\item Se ingresan valores de prueba al modelo y se prueba la clasificación.
\end{itemize}

En investigaciones anteriores a la presente como la de Misigo y Miriti (2016) en la cual se clasificaron
manzanas según su variedad, queda en evidencia que el uso de clasificadores
\textit{Bayes Naive} posee un porcentaje de exactitud mayor sobre componentes de análisis
comparadas con técnicas como la lógica difusa y la red neuronal artificial perceptrón
multicapa, por lo tanto la propuesta de esta investigación fue implementar un clasificador
de \textit{Bayes Naive} para entrenar un modelo que teniendo como características o
entradas los pesos frescos de cada calibre de papa (el calibre es una categorización
de las papas según su tamaño) bajo la hipótesis de que lo lograra hacer con un alto nivel
de aciertos en cuanto a la densidad de siembra en la que fueron plantadas las plantas
de papa criolla. El desarrollo de este clasificador fue en Python que según la fundación
de Python (2018) es un lenguaje de programación creado en 1990 por Guido Van Rossum que
se caracteriza por ser sencillo de aprender, tener estructuras de data de alto nivel que
son eficientes y un simple pero efectivo acercamiento con la programación orientada a
objetos. Este lenguaje actualmente es muy usado en el mundo de la ciencia porque provee
herramientas que facilitan a las personas enfocarse más en el problema que buscan
resolver y no en como programarlo, un ejemplo de esto son los paquetes NumPy (Python
Númerico) y SciPy (Python Científico) (Challenger-Pérez \textit{et al}, 2014).\\

La clasificación de las papas usualmente se realiza de manera manual, existen algunos
modelos de clasificación generados por métodos estadísticos no espaciales que incurren
en la violación de sus supuestos para obtener un resultado y por lo tanto su uso es poco.
Al implementar un clasificador Bayes Naive estos supuestos desaparecen y se puede
realizar una clasificación acertada que pueda ser considerada como un reemplazo a la
clasificación manual. Según la "`Norma NTC 341. Industria Alimentaria - Papa para
Consumo"' la clasificación de tubérculos según su diámetro se realiza bajo las clases
denominadas muy grande (mayores a 90mm), grande (65-90mm), mediana (45-64mm) y pequeña
(30-44mm). Por otra parte la clasificación comercial en Colombia se realiza según grados
(calibres), donde el grado 0 corresponde a tubérculos con un diámetro mayor a 90mm, el
grado 1 a diámetros entre 70 y 89mm, el grado 2 entre 50 y 69mm y el grado 3 entre 35 y
49mm (Buitrago et al, 2003). Sin embargo, los datos para esta investigación siguen la
clasificación de Bernal (2017) quien clasificó los tubérculos según su diámetro en las
categorías menores a 2cm, (2-4]cm, (4-6]cm y mayores de 6cm, estas categorías son
denominadas calibres.\\

Los datos que fueron empleados para la construcción, validación y prueba del clasificador
fueron los datos observados en el Centro agropecuario Marengo de la Universidad Nacional de
Colombia, en el departamento de Cundinamarca.\\

Para realizar la evaluación de los resultados obtenidos para el conjunto de datos
se empleó la curva ROC, que como explica Gönen (2017) la curva ROC (\textit{Receiver
Operating Characteristic Curve}) es una herramienta estadística para evaluar la precisión
de predicciones independientemente de la fuente de las mismas.

\section{Objetivos}

\subsection{Objetivo general}

Diseñar un clasificador de tubérculos de papa criolla para diferentes densidades de
siembra según el peso fresco por calibre empleando \textit{Bayes Naive}.

\subsection{Objetivos espec\'ificos}

\begin{itemize}
\item  Diagnosticar el formato de las variables de entrada y salida para el clasificador.
\item	 Establecer el tipo de algoritmo \textit{Bayes Naive} a emplear y las
características para el entrenamiento.
\item  Implementar el algoritmo de clasificación basado en \textit{Bayes Naive}.
\item  Realizar las pruebas de funcionamiento y la comparación estadística.
\end{itemize}

\section{Aportes de la investigación}

Para la industria colombiana la papa criolla constituye un rubro muy importante, es un
producto versátil que como lo indica Piñeros (2009) permite mediante varios procesos la
obtención de papa criolla precocida y congelada, francesa precocida prefrita congelada
y preformados, entre otros, para la realización de cada uno de estos productos es
necesario emplear papas cuyas características se adecuen a lo que busca la industria,
una característica resaltante al momento de clasificar papas es su peso. Cuando una
industria cosecha papas controla que el resultado sean papas que se adecuen al tipo de
producto que luego van a generar, para ello usan diferentes métodos que les permiten
controlar las características incurriendo en los menores gastos posibles.\\

Para realizar clasificaciones de papa criolla que se basen en la relación que hay entre
la densidad de siembra y los pesos frescos de los calibres de papa no existe un método
o sistema que realice dicha clasificación de manera asertiva, incluso, no existen muchos
métodos que empleen algoritmos que sean capaces de realizar la identificación con gran precisión
empleando estadística, por eso se planteó realizar el clasificador empleando
\textit{Bayes Naive} y Python, \textit{Bayes Naive} por ser un algoritmo de fácil
implementación que ha demostrado en investigaciones pasadas tener un alto nivel de
aciertos y no depende del supuesto de independencia, y se empleó Python como
lenguaje de programación por ser un lenguaje que posee una sintaxis simple, un alto
nivel de efectividad y librerías dedicadas al desarrollo científico, lo que permitió
que el algoritmo que se diseñó sea de fácil uso para personas del campo agronómico, es
decir, es una herramienta base para investigadores que quieran emplear sus
propios clasificadores y evaluarlos con curvas ROC pero no tengan un alto nivel de instrucción
en la parte informática.



