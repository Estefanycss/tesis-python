\chapter{Desarrollo} 

En este capítulo se describe de manera detallada el desarrollo del clasificador de tubérculos siguiendo la metodología descrita en el capítulo 3.
\noindent	
\section{Comprensión de datos}

\noindent
\textbf{Recolección de datos iniciales.}\\

Los datos empleados para realizar la clasificación fueron previamente recolectados por Bernal (2017). En un estudio que se realizó en el Centro agropecuario Marengo de la Universidad Nacional de Colombia, en el departamento de Cundinamarca (74°12'58.51 W; 4°40'52.92 N), el cual tiene una altitud de 2516 msnm, temperatura media de $14^\circ$C  en un rango de $12^\circ$C  a $18^\circ$C  y precipitación media de 500 a 1000 mm, cuenta con un paisaje en planicie fluvio-lacustre y un relieve en terraza lacustre plana (que no excede al 1\%) con suelos moderadamente profundos y bien drenados. El régimen de humedad es ústico y un nivel freático a menos de 0.5m del 15\%. De acuerdo a las características de precipitación, temperatura y evapotranspiración, la zona se clasifica como Bosque Seco Montano Bajo. (Bernal, 2017)\\

El material vegetal utilizado corresponde al cultivo de papa criolla Solanum phureja, utilizando el tubérculo como semilla con el tamaño y forma característica de la especie (tamaño mediano), ojos poco profundos, sin pudrición ni defectos en la piel. Esta variedad con un porte de planta medio y follaje verde claro, distinguida por su adaptación a días cortos, de origen y distribución en América del Sur, y con centro de diversidad genética al sur de Colombia. Con un desarrollo vegetativo que se da hasta los 35 días después de la siembra (dds), siguiendo la oración hasta los 65 dds, fructificación hasta los 90 dds y finalmente la madurez y senescencia hasta los 120 dds. Esta variedad es precoz (120 días a 2600 msnm), su potencial de rendimiento en condiciones óptimas de cultivo es de 15 a 25 ton.ha−1, sin periodo de reposo y susceptible al virus del amarillamiento de las nervaduras de la hoja (Potato yellow vein virus). Se cultiva en las diferentes regiones de Colombia y en diferentes condiciones de suelo. Es la principal variedad de papa criolla cultivada en el país y hasta la presente es la variedad que se procesa para exportación como precocida congelada (Ñustez, 2011; Rodríguez y Ñustez, 2011).\\

A los 120 dds se cosecharon los tubérculos y se contaron según su diámetro en las categorías 2 cm, (2-4] cm, (4-6] cm y > 6 cm. Para estudiar el efecto de la densidad de siembra se fijaron las distancias entre plantas de (30, 40 y 50 cm) todos con separación de 100 cm entre surcos. La siembra se realizó en surcos alineados con precisión según la densidad de siembra, utilizando tres surcos sucesivos según la geometría del lote para cada densidad, con dos repeticiones por densidad, lo que rindió un total de 18 surcos, para un total de 2841 plantas. Aunque la unidad que aportó cada dato fue la planta (tubérculos), la obvia dificultad para aleatorizar una densidad de siembra usando cada planta como unidad experimental, obligó a la aleatorización de las densidades de siembra, cada una con sus tres respectivos surcos (unidad experimental) dentro del lote, registrando los datos de cada planta (unidad de observación). Bajo estas condiciones, el diseño resultó ser una factorial simple en arreglo completamente al azar, tomando las distancias entre plantas como los niveles del factor. (Bernal, 2017)\\

\noindent
\textbf{Descripción de los datos.}\\

Los datos se encuentran en un archivo de Excel (.xlsl) ordenados por columnas, la primera denominada planta que contiene el número que identifica cada planta, la segunda columna llamada densidad que posee valores de 1, 2 o 3, donde 1 implica una densidad de siembra de 30cm, 2 una densidad de 40cm y 3 una densidad de 50cm. Las siguientes cuatro columnas estan identificadas como PD1, PD2, PD3 y PD4, cada columna posee valores continuos que representan el peso fresco de cada calibre en esa planta, los calibres son 4 y representan la categorización de los tubérculos según su diámetro, para el calibre 1 se toman tubérculos con diámetro menor o igual a 2cm, para el calibre 2 con diámetro mayor a 2cm y menor o igual a 4cm, en el calibre 3 mayores a 4cm y menores o igual a 6cm y finalmente el calibre 4 con tubérculos de diámetro mayor a 6cm. Las últimas dos columnas son X y Y, que indican mediante un valor entero la posición de dicha planta en la siembra. La cantidad total de plantas es 2839, para las densidades de siembra 1, 2 y 3 hay 1135, 926 y 778 plantas respectivamente, cada una con valores de PD1, PD2, PD3, PD4, X y Y.

A continuación se observan los datos descriptivos de cada variable

\begin{table}[htbp]
\begin{center}
\begin{tabular}{|l|l|l|l|l|l|}
\hline
& Densidad & Peso Calibre 1 & Peso Calibre 2 & Peso Calibre 3 & Peso Calibre 4  \\
\hline \hline
Conteo & 2839 & 2839 & 2839 & 2839 & 2839  \\ \hline
Media & 1.8742 & 74.1536gr & 226.3262gr & 213.3114gr & 18.9698gr  \\ \hline
Mínimo & 1 & 0gr & 0gr & 0gr & 0gr  \\ \hline
Máximo & 3 & 390.0000gr & 1080.0000gr & 1290.0000gr & 775.9459gr  \\ \hline
Mediana & 2 & 62.0547gr &195.0000gr & 164.7058 & 0.0000gr \\ \hline
Varianza & 0.6582 & 2892.7971 & 26386.3844 & 36986.5868 & 3486.4316 \\ \hline
Moda & 0 & 0gr & 0gr & 0gr & 0gr  \\ \hline
Distribución & N/A & normal & normal & normal & normal  \\ \hline
\end{tabular}
\caption{Datos Estadísticos Descriptivos.}
\label{tabla:sencilla}
\end{center}
\end{table}

\noindent
\textbf{Exploración de datos.}\\

\noindent
\textbf{Verificación de la calidad de los datos.}\\

\section{Preparación de datos}

\noindent
\textbf{Selección de datos.}\\

\noindent
\textbf{Limpieza de los datos.}\\

\noindent
\textbf{Estructuración de los datos.}\\

\noindent
\textbf{Integración de los datos.}\\

\noindent
\textbf{Formateo de los datos.}\\

\section{Modelado}

\noindent
\textbf{Técnica de modelado.}\\

\noindent
\textbf{Método de evaluación de los modelos.}\\

\section{Evaluación}

\noindent
\textbf{Evaluación de los resultados.}\\

\noindent
\textbf{Proceso de revisión.}\\

\noindent
\textbf{Determinación de futuras fases.}\\